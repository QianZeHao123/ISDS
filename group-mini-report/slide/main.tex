% !TEX TS-program = pdflatex
% !TEX encoding = UTF-8 Unicode
% 
% This file is a template using the "beamer" package to create slides for a talk or presentation
% - Talk at a conference/colloquium.
% - Talk length is about 20min.
% - Style is ornate.
% 
% MODIFIED by Jonathan Kew, 2008-07-06
% The header comments and encoding in this file were modified for inclusion with TeXworks.
% The content is otherwise unchanged from the original distributed with the beamer package.
\documentclass{beamer}
\input{header.tex}
\input{information.tex}
% 
% If you wish to uncover everything in a step-wise fashion, uncomment
% the following command: 
%\beamerdefaultoverlayspecification{<+->}
\begin{document}
% 
\begin{frame}
  \titlepage
\end{frame}
% \input{CV.tex}
% 
\begin{frame}
  \tableofcontents
  % You might wish to add the option [pausesections]
\end{frame}
% 
% 
% Structuring a talk is a difficult task and the following structure
% may not be suitable. Here are some rules that apply for this
% solution: 
% 
% - Exactly two or three sections (other than the summary).
% - At *most* three subsections per section.
% - Talk about 30s to 2min per frame. So there should be between about
%   15 and 30 frames, all told.
% 
% - A conference audience is likely to know very little of what you
%   are going to talk about. So *simplify*!
% - In a 20min talk, getting the main ideas across is hard
%   enough. Leave out details, even if it means being less precise than
%   you think necessary.
% - If you omit details that are vital to the proof/implementation,
%   just say so once. Everybody will be happy with that.
% 
% 
\section{Introduction}
\subsection{Intro to Data Set and its Context}
% 
% 
% 
% 
% 
% 
\begin{frame}
  \frametitle{What is Happiness?}
  \begin{figure}
    \includegraphics[width=\textwidth]{img/What is happyness.png}
  \end{figure}
\end{frame}
% 
% 
% 
% 
\begin{frame}
  \frametitle{Data Set and its Context}
  \begin{figure}
    \includegraphics[width=\textwidth]{img/Intro to Dataset.png}
  \end{figure}
\end{frame}
% 
% 
% 
% 
% 
% 
% 
% 
% 
% 
% 
% 
\section{Statistical Modeling}
% 
% 
% 
% 
\subsection{Model Explanation}
% 
% 
% 
% 
\begin{frame}
  \frametitle{Multiple Linear Regression Model}
  \begin{itemize}
    \item \textbf{Model Target:} Some socio-economic indexes are used to predict Ladder Score to assist government decision-making.
    \item \textbf{Independent Variables:} LGDP, Support, HLE, Freedom, Corruption, Continent
    \item \textbf{Dependent Variables:} Ladder Score
    \item \textbf{Model Function:} $input(LGDP, Support,...) \Rightarrow output(Ladder\ Score)$
          $ Ladder Score = \beta_0 + \beta_1 * LGDP + \beta_2 * Support + ... + \epsilon $
    \item \textbf{Optimization Target:} Making the model with the \textcolor{red}{best subset} and \textcolor{red}{higher Adjusted R-squared}. $
            \left\{
            \begin{matrix}
              Select\ the\ independent\ variables \\
              Update\ \beta\ and\ \epsilon\ to\ minimize\ the\ residual\ sum\ of\ squared
            \end{matrix}
            \right.
          $
  \end{itemize}
  % 
  % 
  % 
  % 
\end{frame}
% 
% 
% 
% 
% \begin{frame}{Comparison of Linear Regression Models}
\begin{frame}{Why we choose Multiple Linear Regression Model}
  \begin{figure}
    \includegraphics[width=\textwidth]{img/Why choose this model.png}
  \end{figure}
\end{frame}
% 
% 
% 
% 
\subsection{Model Construction}
% 
% 
% 
% 
\begin{frame}
  \frametitle{Constructing the Model -- Data Processing}
  \begin{figure}
    \includegraphics[width=\textwidth]{img/Data Process.png}
  \end{figure}
\end{frame}
% 
% 
% 
% 
\begin{frame}
  \frametitle{Constructing the Model -- Dataset Inspectation}
  \begin{figure}
    \includegraphics[width=\textwidth]{img/Dataset We have.png}
  \end{figure}
\end{frame}
% 
% 
% 
% 
% 
\begin{frame}
  \frametitle{Constructing the Model -- Bring dataset to MLR model}
  \begin{figure}
    \includegraphics[width=\textwidth]{img/UsingMLR.png}
  \end{figure}
\end{frame}
% 
% 
% 
% 
% 
% 
% 
% 
% 
% 
\section{Conclusions and Analysis}
% 
% 
% 
% 
\subsection{Results and Analysis}
% 
% 
% 
% 
\begin{frame}
  \frametitle{Analysing the model: Simple Linear Regression}
  \framesubtitle{Check Assumptions:}
  \begin{columns}[onlytextwidth]
    \begin{column}{0.5\textwidth}
      \begin{enumerate}
        \item Mean = 0
        \item Homoskedasticity
      \end{enumerate}
    \end{column}
    \begin{column}{0.5\textwidth}
      \begin{enumerate}
        \setcounter{enumi}{2}
        \item Independence
        \item Normally Distributed
      \end{enumerate}
    \end{column}
  \end{columns}
  % 
  % 
  % 
  % 
  \begin{figure}
    \includegraphics[width=\textwidth]{img/Assumption Check.png}
  \end{figure}
\end{frame}
% 
% 
% 
% 
% 
% 
% 
% 
\begin{frame}
  \frametitle{Analysing the model: Multiple Linear Regression}
  \begin{itemize}
    \item VIF to check if there is a correlation between predictor variables;
    \item Example: \begin{itemize}
            \item Africa shows low values
            \item Europe shows more of a spread
          \end{itemize}
  \end{itemize}
  % 
  % 
  % 
  % 
  \begin{figure}
    \includegraphics[width=0.8\textwidth]{img/Vif Check.png}
  \end{figure}
  % 
  % 
  % 
  % 
\end{frame}
% 
% 
% 
% 
% 
% 
% 
% 
% 
\begin{frame}
  \frametitle{Best Subset Selection}
  \begin{itemize}
    \item Europe shows agreement between the range of models.
    \item Africa shows a larger range of possible values with far more variable graphical shapes.
  \end{itemize}
  % 
  \begin{figure}
    \includegraphics[width=\textwidth]{img/Best Subset Selection.png}
  \end{figure}
  % 
  % 
\end{frame}
% 
% 
% 
% 
\subsection{Conclusions}
% 
% 
% 
% 
\begin{frame}
  \frametitle{Correlation}
  % 
  % 
  % 
  %
  \begin{figure}
    \includegraphics[width=\textwidth]{img/Scatter plot.png}
  \end{figure}
\end{frame}
% 
% 
% 
% 
% 
% 
\begin{frame}
  \frametitle{Equations}
  % 
  % 
  \begin{figure}
    \includegraphics[width=\textwidth]{img/Equations.png}
  \end{figure}
  % 
  % 
\end{frame}
% 
% 
% 
% 
% 
% 
% 
\begin{frame}
  \frametitle{Government Advice}
  % 
  \begin{itemize}
    \item By looking at the general, overall dataset clear support is the most important metric and so the variable to focus on improving first. \\ \ \\ \ \\
    \item However, through best subset selection it is clear that a range of variables all come into this, and so focus on purely support will no yield the best results, although it should be prioritised.
  \end{itemize}
\end{frame}
% 
% 
% 
% 
% 
% 
% 
% 
\begin{frame}
  \frametitle{Advice per Continent}
  % 
  For each continent, there are certain differences that stood out and should be highlighted for governments within them: \\ \
  \begin{enumerate}
    \item Africa: LGDP is the most important metric and to such an extent that ladder score can almost be modelled directly off it. Thus to improve happiness, LGDP is key to focus on;
    \item Europe: Similar to the general dataset and shows a range of factors will need to be focused on, starting with LGDP;
    \item Asia: Focus on all factors, however with a prioritisation of  LGDP, Support and Freedom;
    \item North America: A focus on LGDP and Freedom is crucial to improving the Happiness extent;
    \item Data for South America and Oceania is not sufficient to currently draw conclusions from.
  \end{enumerate}
\end{frame}
% 
% 
% 
% 
% 
% 
\section{Future Work}
% 
% 
% 
% 
\begin{frame}
  \frametitle{Next Steps}
  \begin{enumerate}
    \item \textbf{Improve the accuracy of the model}
          \begin{itemize}
            \item Interaction between variables (e.g. $\textit{HLE} \times \textit{Support}$).
            \item Try non-linear transformation of the predictors (e.g. $log(HLE)$, $HLE^2$, $HLE^3$).
          \end{itemize}
    \item \textbf{Bug fixing}
          \begin{itemize}
            \item The data for North America was anomalous and independent variables could not be selected by linear regression analysis. Our team will next process anomalous data from the North American dataset.
            \item There are only two countries in Oceania, and when they are brought into the analytical model, they report an error and cannot perform linear regression. We will try to combine the Oceania data into other similar continents for analysis.
          \end{itemize}
  \end{enumerate}
\end{frame}
% 
% 
% 
% 
% 
% 
% 
% 
% 
% 
% 
% 
% 
% 
\begin{frame}
  \frametitle{}
  \centering
  \vfill
  \Huge Thanks for watching! \\
  \vfill
  \begin{figure}
    \includegraphics[width=0.4\textwidth]{img/qrcode_github.com.png}
  \end{figure}
  \vfill
  \small ISDS Group 10 \\
  \today
\end{frame}
% 
% 
% 
% 
% 
% 
% 
% 
% 
% 
% 
% 
% 
% 
% \section{Introduction}
% \subsection{Background}
% \begin{frame}
%   \frametitle{The Premier League}
%   \begin{itemize}
%     \item Premier League: Top tier of English Football League System.
%     \item 20 teams play 38 home and away matches.
%     \item Globally renowned and challenging to predict outcomes.
%   \end{itemize}

%   \textbf{Background}

%   \begin{enumerate}
%     \item Outcome predictions involve expert analysis.
%     \item Factors include team performance, player form, and tactics.
%     \item Growing data, e.g., player touches, team running stats, manager experience.
%   \end{enumerate}
% \end{frame}

% % All of the following is optional and typically not needed. 
% % 
% % 
% % 
% % 
% % ----------------------------------------------------------------------
% \section{Mathematical Modeling}
% % 
% % 
% % 
% % 
% \subsection{Method 1: Entropy Weight Method in Football Team Evaluation}
% \begin{frame}
%   \frametitle{Overview of Entropy Weight Method in Football}
%   % 
%   % 
%   % 
%   % 
%   % 
%   % 
%   \begin{enumerate}
%     \item Introduction
%           \begin{itemize}
%             \item The Entropy Weight Method is a powerful analytical technique used in football team evaluation. It goes beyond traditional methods by considering the inherent information entropy within various performance attributes.
%           \end{itemize}
%     \item Key Characteristics
%           \begin{itemize}
%             \item Entropy: Reflects the degree of uncertainty or randomness within a dataset.
%             \item Weight Assignment: Assigns weights to attributes based on their information entropy.

%           \end{itemize}
%     \item Objective
%           \begin{itemize}
%             \item The method aims to provide a nuanced evaluation, giving higher importance to attributes that contribute more to understanding a team's performance.
%           \end{itemize}
%   \end{enumerate}


% \end{frame}
% % 
% % 
% % 
% % 
% \begin{frame}
%   \frametitle{Key Steps in Entropy Method}
%   \begin{enumerate}
%     \item \textbf{Data Collection and Attribute Selection}
%     \item \textbf{Entropy Calculation:}
%           \begin{itemize}
%             \item Utilize mathematical formulas to calculate the entropy of each selected attribute.
%             \item Entropy = $- \sum (p_i \cdot \log_2(p_i))$, where $p_i$ is the probability of each attribute value.
%           \end{itemize}

%     \item \textbf{Weight Assignment:}
%           \begin{itemize}
%             \item Assign weights to attributes based on their calculated entropy.
%             \item Attributes with higher entropy receive lower weights, and vice versa.
%             \item The sum of weights equals 1 for normalization.
%           \end{itemize}

%     \item \textbf{Outcome:}
%           \begin{itemize}
%             \item The result is a set of weights that reflect the relative importance of each attribute in evaluating a football team's performance.
%           \end{itemize}
%   \end{enumerate}
% \end{frame}
% % 
% % 
% \begin{frame}
%   \frametitle{Entropy Method in Our Model}
%   \begin{figure}
%     \includegraphics[width=\textwidth]{img/entropy_method_flowchart.png}
%   \end{figure}
%   % 
%   % 
% \end{frame}
% % 
% % 
% % 
% % 
% % 
% %  
% % 
% % 
% % 
% % 
% % 
% % 
% % 
% % 
% \subsection{Method 2: Linear Transformation and Gradient Ascend}
% \begin{frame}
%   \frametitle{Linear Transformation and Gradient Ascend}
%   \begin{enumerate}
%     \item Linear transformation involves transforming input variables into a new space where they are more linear and easier to model
%     \item Our model will be taking in 5-dimension vector and transforming it to a 1-dimension vector $$ F_v: \mathbb{R}^{5} \Rightarrow \mathbb{R} $$
%     \item The vector v basically consists of the following variables:
%           \begin{itemize}
%             \item The number of games won by a team in $n^{th}$ week, $v_1$
%             \item The number of games lost by a team in $n^{th}$ week, $v_2$
%             \item The number of games drawn by a team in $n^{th}$ week, $v_3$
%             \item The goal difference (goals scored - goals conceded) of a team in $n^{th}$ week, $v_4$
%             \item The number of points scored by a team in $n^{th}$ week, $v_5$
%           \end{itemize}
%   \end{enumerate}
% \end{frame}
% % 
% % 
% % 
% % 
% % 
% % 
% % 
% % 
% \begin{frame}
%   \frametitle{Linear Transformation and Gradient Ascend}
%   \begin{itemize}
%     \item The linear transformation will transform the 5-dimensional vector into a one-dimensional vector which is the total number of points in the final week $F(v)$
%     \item To calculate we will use the following formula: $$ F(v)=wv+b $$
%     \item $w$ is a 5-dimensional vector, weights of each variable in vector v
%     \item $b$ is the scalar bias - systematic error

%   \end{itemize}
% \end{frame}
% % 
% % 
% % 
% % 
% % 
% \begin{frame}
%   \frametitle{Using Gradient Ascend}
%   \begin{itemize}
%     \item To calculate w and b, we will use gradient ascend on historical data.
%     \item Normalize $v$ and $F(v)$ by centering and standardizing it.
%     \item Use normalized data and take out predicted points using any initial value of weight.
%     \item See the error by finding the difference between normalized $F(v)$ and predicted $F(v)$
%     \item Find the gradient of the weight using the formula - transposed-v.error
%     \item Find the gradient of scalar bias using the formula - sum of all elements in error vector
%     \item Then to find the updated gradient the following formula is:
%     $$ w = w + a*w_{gradient}, \ b = b + a*b_{gradient} $$
%     \item Then denormalize w and b to find the actual values

%   \end{itemize}
% \end{frame}
% % 
% % 
% % 
% % 
% \section{Result Analysis}
% \subsection{Grabbing Data from Web}
% \begin{frame}
%   \frametitle{Grabbing Data from Web}
%   \begin{figure}
%     \includegraphics[width=\textwidth]{img/DataGrab.png}
%   \end{figure}
% \end{frame}
% \subsection{Evaluating the Teams's Tier}
% \begin{frame}
%   \frametitle{Evaluating the Teams's Tier with Entropy Method}
%   \begin{figure}
%     \includegraphics[width=\textwidth]{img/Entropy_output.png}
%   \end{figure}
% \end{frame}
% % 
% % 
% % 
% % Prediction of 2023/24's final point
% % 
% % 
% % 
% \subsection{Prediction of 2023/24's winner point}
% \begin{frame}
%   \frametitle{Prediction of 2023/24's winner point}
%   \begin{figure}
%     \includegraphics[width=\textwidth]{img/prediction_current_year.png}
%   \end{figure}
% \end{frame}
% % 
% % 
% % 
% % 
% % 
% % 
% % 
% % 
% \section{Future Work}
% \subsection{Our plans, our limitations and how these models could improve}
% \begin{frame}
%   \frametitle{Future plans}
%   \begin{itemize}
%     \item Once our models are all complete, we will compare them with respect to efficiency and accuracy in predicting the winning team with this year's data as input. We could make use of Mallows' Cp and BIC statistic to establish which model has the best predictive performance.
%     \item We will then critically analyse the most successful model and input a test dataset from earlier years to ensure we have not overfitted to our training dataset.
%     \item To improve this model, it would be wise to consider variables other than the teams' performance and establish whether they are confounding and extraneous; for example, the about of money invested in each team.
%   \end{itemize}
% \end{frame}
% 
% 
% 
% 
% 
% 
% 

% \begin{frame}
%   \frametitle{What is this model useful for}
%   \begin{enumerate}
%     \item Once the most significant variables have been established, a researcher could potentially use the model to advise  a team's manager of where best to focus their efforts, or indeed they could consult with a bookie to help them predict the odds of the game.
%     \item Of course, this model overlooks the human element of the game. Factors that a manager might be aware of such as player motivation, injuries, and team dynamics can have a significant impact on the outcome of a game, but they are challenging to factor into our model.
%     \item Ultimately though, it is impossible to control for every variable and, even if we did the result could surprise us. But this model is beneficial in finding which variables give the teams the best chance of success.
%   \end{enumerate}
% \end{frame}
% 
% 
% 
% 
% 
% 
% 
% 
\end{document}
