\title[] % (optional, use only with long paper titles)
{Introduction to Statistic for Data Science}

\subtitle
{Group Mini-Project Presentation: Happiness Ladder}

\author % (optional, use only with lots of authors)
{ISDS Group 10}
% - Give the names in the same order as the appear in the paper.
% - Use the \inst{?} command only if the authors have different
%   affiliation.

\institute[Data Science] % (optional, but mostly needed)
{
  Christopher Barrow, Zehao Qian, Mengyuan Zhu, Hithein Augustine \\ \ \\
  Department of Natural Sciences \\
  Durham University, England, UK
}
% - Use the \inst command only if there are several affiliations.
% - Keep it simple, no one is interested in your street address.

% \date[CFP 2003] % (optional, should be abbreviation of conference name)
% {Conference on Fabulous Presentations, 2003}
% - Either use conference name or its abbreviation.
% - Not really informative to the audience, more for people (including
%   yourself) who are reading the slides online
% \subject{Theoretical Computer Science}
% This is only inserted into the PDF information catalog. Can be left
% out. 
% If you have a file called "university-logo-filename.xxx", where xxx
% is a graphic format that can be processed by latex or pdflatex,
% resp., then you can add a logo as follows:
% \pgfdeclareimage[height=0.5cm]{university-logo}{university-logo-filename}
% \logo{\pgfuseimage{university-logo}}
% Delete this, if you do not want the table of contents to pop up at
% the beginning of each subsection:
\AtBeginSubsection[]
{
  \begin{frame}<beamer>{Outline}
    \tableofcontents[currentsection,currentsubsection]
  \end{frame}
}