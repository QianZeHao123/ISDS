% !TEX TS-program = pdflatex
% !TEX encoding = UTF-8 Unicode
% 
% This is a simple template for a LaTeX document using the "article" class.
% See "book", "report", "letter" for other types of document.
% 
\documentclass[11pt]{article} % use larger type; default would be 10pt
% 
% 
% 
%%% The "real" document content comes below...
\usepackage[utf8]{inputenc} % set input encoding (not needed with XeLaTeX)

%%% Examples of Article customizations
% These packages are optional, depending whether you want the features they provide.
% See the LaTeX Companion or other references for full information.

%%% PAGE DIMENSIONS
\usepackage{geometry} % to change the page dimensions
\geometry{a4paper} % or letterpaper (US) or a5paper or....
% \geometry{margin=2in} % for example, change the margins to 2 inches all round
% \geometry{landscape} % set up the page for landscape
%   read geometry.pdf for detailed page layout information

\usepackage{graphicx} % support the \includegraphics command and options

% \usepackage[parfill]{parskip} % Activate to begin paragraphs with an empty line rather than an indent

%%% PACKAGES
\usepackage{booktabs} % for much better looking tables
\usepackage{array} % for better arrays (eg matrices) in maths
\usepackage{paralist} % very flexible & customisable lists (eg. enumerate/itemize, etc.)
\usepackage{verbatim} % adds environment for commenting out blocks of text & for better verbatim
\usepackage{subfig} % make it possible to include more than one captioned figure/table in a single float
% These packages are all incorporated in the memoir class to one degree or another...

%%% HEADERS & FOOTERS
\usepackage{fancyhdr} % This should be set AFTER setting up the page geometry
\pagestyle{fancy} % options: empty , plain , fancy
\renewcommand{\headrulewidth}{0pt} % customise the layout...
\lhead{}\chead{}\rhead{}
\lfoot{}\cfoot{\thepage}\rfoot{}

%%% SECTION TITLE APPEARANCE
\usepackage{sectsty}
\allsectionsfont{\sffamily\mdseries\upshape} % (See the fntguide.pdf for font help)
% (This matches ConTeXt defaults)

%%% ToC (table of contents) APPEARANCE
\usepackage[nottoc,notlof,notlot]{tocbibind} % Put the bibliography in the ToC
\usepackage[titles,subfigure]{tocloft} % Alter the style of the Table of Contents
\renewcommand{\cftsecfont}{\rmfamily\mdseries\upshape}
\renewcommand{\cftsecpagefont}{\rmfamily\mdseries\upshape} % No bold!

%%% END Article customizations
\usepackage{xcolor}
\usepackage{tcolorbox}
\usepackage{lipsum}  % 示例文本
\usepackage{mdframed}
\usepackage{pdfpages}


% R code support
\usepackage{listings}
% \lstset{language=R,  % 设置语言为R
%         basicstyle=\ttfamily, % 设置字体样式
%         numbers=left,  % 行号显示在左侧
%         numberstyle=\small\color{blue},  % 行号样式
%         frame=single,  % 设置代码块的边框
%         backgroundcolor=\color{lightgray},  % 设置代码块的背景颜色
%         }

\usepackage{graphicx}
\usepackage{float}
\usepackage{amsmath}





% 设置R语言代码高亮
\lstdefinestyle{rstyle}{
  language=R,
  basicstyle=\ttfamily,
  numbers=left,
  numberstyle=\tiny\color{gray},
  commentstyle=\color{green!40!black},
  keywordstyle=\color{blue},
  stringstyle=\color{purple},
  morekeywords={read.csv},
  frame=single,
  breaklines=true
}
% 
% 
\title{Introduction to Statistics for Data Science \\ Hypothesis Testing}
\author{Zehao Qian}
\begin{document}
\maketitle
% 
% 
% \input{example}
\section{Recall: Normal Distribution}
\subsection{Learning Materials}
\begin{enumerate}
    \item How reliable are the values found?
    \item How reliable are the conclusions drawn?
\end{enumerate}
\subsection{Learning Objectives}
\begin{enumerate}
    \item Define the chi-squared distribution
    \item Define the T-distribution
    \item Define the F-distribution
\end{enumerate}
% 

\subsection{Recall: The Normal Distribution}
\begin{itemize}
    \item Denote by $X \sim N(\mu, \sigma^2)$
    \item $E(X)=\mu$, $Var(X)=\sigma^2$
    \item $$ f(X)=\frac{1}{\sigma \sqrt{2 \pi}}e^{-\frac{1}{2}(\frac{x-\mu}{\sigma})^2} $$
    \item Standard normal:$Z \sim N(0,1)$
    \item $E(Z)=0$, $Var(Z)=1$
    \item $$ \frac{X-\mu}{\sigma}=Z \mathbf{, Standardisation}$$
\end{itemize}
% 
% 
% 
% 
% 
\subsection{Chi-Squared Distribution}
\paragraph{AKA Chi-square distribution, AKA $\chi^2$ Distribution}
% 
% 
% 
\begin{itemize}
    \item Resuqired one parameter,natural number k
    \item $$ U_k \sim \chi_k^2 $$
    \item $Z_1$, $Z_2$, $Z_3$,... ,$Z_k$: k independent standard normal random variables
    \item $$ U_k = \sum_{i=1}^{k}Z_i^2 $$
\end{itemize}
% 
% 
\subsubsection{The degree of Freedom}
\paragraph{Conception: The number of independent values available to us when calculate a value}
\paragraph{For $U_k$ we need k independent values,$Z_1$, $Z_2$, $Z_3$,... ,$Z_k$ hense k degree of freedom}
% 
% 
% 
% 
% 
% 
% 
% 
\subsection{t-Distribution}
\paragraph{AKA T distribution, AKA stident's t-Distribution}
% 
% 
\begin{itemize}
    \item Resuqired one parameter, degrees of freedom k
    \item $$ T_k \sim t_k $$
    \item $$ T_k = \frac{Z}{\sqrt{\frac{U_k}{k}}} $$
\end{itemize}
% 
% 
% 
% 
% 
% 
% 
\subsection{F-Distribution}
% \paragraph{AKA T distribution, AKA stident's t-Distribution}
% 
% 
\begin{itemize}
    \item Resuqired two parameters, degrees of freedom m and n
    \item $$ F_{m,n} \sim F_{m,n} $$
    \item $$ F_{m,n}=\frac{U_m}{U_n} $$
\end{itemize}
% 
% 
% 
% 
% 
% 
% 
% 
% 
% 
% 
% 
% 
% 
% 
% 
% 
% 
% 
% 
% 
% 
% 
% 
% 
% 
% 
% 
% 
% 
% 
% 
\section{Sample}
\paragraph{What is the use of a sample?}
\begin{itemize}
    \item Can use to calculate statistics.
    \item Hope those statistics have values close to true parameter of population.
\end{itemize}
% 
% 
% 
\subsection{Learning Objectives}
\begin{enumerate}
    \item Get familiar with various aproaches to sampling
    \item Define and make use of sampling distribution
    \item Exress and make use of the \textbf{Central Limit Theorem}
\end{enumerate}
% 
% 
% 
% 
% 
\subsection{Why sampling?}
\subsection{Types of sampling}
\begin{itemize}
    \item simple random sampling
    \item stratified sampling
    \item cluster sampling
    \item multistage sampling
    \item Non-random sampling
          \begin{itemize}
              \item Convenience sampling
              \item Judgement sampling
              \item Quota sampling
          \end{itemize}
\end{itemize}
% 
% 
\subsection{The sampling distribution}
\paragraph{Function of random variables}
\begin{itemize}
    \item If $X_1$, $X_2$,... ,$X_n$ are random variables
    \item $$ f(X_1, X_2,... ,X_n)=\frac{\sum_{i=1}^{n}X_i}{n}=Y $$
    \item Y must also be a random variables
    \item Y must also be the mean value, $\bar{X}$
    \item The mean of RVs is also an RV
\end{itemize}
% 
% 
% 
\paragraph{The distribution of a statistic over repeated samples}
\begin{itemize}
    \item proportion $\hat{p}$
    \item Variance $s^2$
    \item $$ E(\bar{X})=\mu_{\bar{X}}=\mu $$
\end{itemize}
% 
% 
% 
% 
\subsection{The Central Limit Theorem (CLT)}
\begin{itemize}
    \item Let X be any random variables, $E(X)=\mu$, $Var(X)=\sigma^2$
    \item Collect n realisation of X to get $\bar{x}$
    \item $$ \mathbf{Set } \bar{X_n}=\sum_{i-1}^{n}\frac{X_i}{n} $$
\end{itemize}
% 
% 
% 
% 
\paragraph{\textbf{Central Limit Theorem: }for a large n, the sampling distribution of $\bar{X_n}$ is approximately $N(\mu, \frac{\sigma^2}{n})$}
% 
$$ lim_{n \rightarrow \infty }  \bar{X_n} \sim N(\mu, \frac{\sigma^2}{n})$$
% 
% 
% 
% 
\subsubsection{Three points about the CLT}
\begin{enumerate}
    \item n should be at least 30
    \item a sum of normal RVs is a normal RV, A normal RV divided by n is a normal RV \\
          Each $X_i$ normal $\Rightarrow$ $\bar{X}$ normal \\
          $\bar{X_n \sim N(\mu, \frac{\sigma^2}{n})}$ true for all value of n
    \item Cab standardise a  normal RV X
    \item $$ lim_{n \rightarrow \infty}\bar{X_n} \sim N(\mu,\frac{\sigma^2}{n}) \Rightarrow \lim_{n \rightarrow \infty} \frac{\bar{X_n}-\mu}{\frac{\sigma}{\sqrt{n}}} \sim N(0,1)$$
\end{enumerate}
% 
% 
% 
\subsubsection{Infinite Population}
$$ \sigma_{\bar{X}}^2 = \frac{\sigma^2}{n} \Leftarrow $$
\paragraph{Makes no difference and ALWAYS TRUE under condition above}
% 
% 
\subsubsection{Finite Population}
\paragraph{Each value sampled changes nature of remaining unsampled popukation.}
$$ \sigma_{\bar{x}}^2=\frac{\sigma^2}{n} \times \frac{N-n}{N-1}$$
\begin{itemize}
    \item N is size of population
    \item Finite population
    \item $\sigma_{\bar{x}}^2$ for finite population with variance $\sigma^2 < \sigma_{\bar{x}}^2$ infinite population with variance $\sigma^2$
    \item As n increases, $\sigma_{\bar{X}}^2$ gets smaller for same N and $\sigma$
\end{itemize}
% 
% 
% 
% 
% 
% 
% 
% 
\subsection{Standard Error}
\subsection{Variance of A Sampling Distribution}
% 
% 
$$ s^2=\frac{1}{n-1} \sum_{i=1}^{n} (x_i-\bar{x})^2 $$
% 
% 
$$ \frac{s^2(n-1)}{\sigma^2} \sim \chi_{n-1}^2 $$
% 
% 
$$ E(s^2) = \frac{\sigma^2}{n-1}E(\chi_{n-1}^2) $$
% 
\paragraph{Due to the fact that $E(\chi_k^2)=k$, so $E(s^2)=\sigma^2$}
% 
% 
% 
% 
% 
% 
% 
% 
% 
% 
% 
% 
% 
% 
% 
% 
% 
% 
% 
% 
% 
% 
% 
% 
% 
% 
% 
% 
% 
% 
% 
\section{Estimation}
% 
% 
\subsection{Types of Estimation}
\begin{itemize}
    \item Point Estimation
    \item Interval Estimation
\end{itemize}
% 
% 
\subsection{Learning Objects}
\begin{enumerate}
    \item Explore
\end{enumerate}
% 
% 
% 
% 
% 
% 
% 
% 
% 
% 
% 
% 
% 
% 
% 
% 
% 
% 
% 
% 
% 
% 
% 
%
\end{document}
% 
% 