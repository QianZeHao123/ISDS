% !TEX TS-program = pdflatex
% !TEX encoding = UTF-8 Unicode

% This is a simple template for a LaTeX document using the "article" class.
% See "book", "report", "letter" for other types of document.

\documentclass[11pt]{article} % use larger type; default would be 10pt
\usepackage[utf8]{inputenc} % set input encoding (not needed with XeLaTeX)

%%% Examples of Article customizations
% These packages are optional, depending whether you want the features they provide.
% See the LaTeX Companion or other references for full information.

%%% PAGE DIMENSIONS
\usepackage{geometry} % to change the page dimensions
\geometry{a4paper} % or letterpaper (US) or a5paper or....
% \geometry{margin=2in} % for example, change the margins to 2 inches all round
% \geometry{landscape} % set up the page for landscape
%   read geometry.pdf for detailed page layout information

\usepackage{graphicx} % support the \includegraphics command and options

% \usepackage[parfill]{parskip} % Activate to begin paragraphs with an empty line rather than an indent

%%% PACKAGES
\usepackage{booktabs} % for much better looking tables
\usepackage{array} % for better arrays (eg matrices) in maths
\usepackage{paralist} % very flexible & customisable lists (eg. enumerate/itemize, etc.)
\usepackage{verbatim} % adds environment for commenting out blocks of text & for better verbatim
\usepackage{subfig} % make it possible to include more than one captioned figure/table in a single float
% These packages are all incorporated in the memoir class to one degree or another...

%%% HEADERS & FOOTERS
\usepackage{fancyhdr} % This should be set AFTER setting up the page geometry
\pagestyle{fancy} % options: empty , plain , fancy
\renewcommand{\headrulewidth}{0pt} % customise the layout...
\lhead{}\chead{}\rhead{}
\lfoot{}\cfoot{\thepage}\rfoot{}

%%% SECTION TITLE APPEARANCE
\usepackage{sectsty}
\allsectionsfont{\sffamily\mdseries\upshape} % (See the fntguide.pdf for font help)
% (This matches ConTeXt defaults)

%%% ToC (table of contents) APPEARANCE
\usepackage[nottoc,notlof,notlot]{tocbibind} % Put the bibliography in the ToC
\usepackage[titles,subfigure]{tocloft} % Alter the style of the Table of Contents
\renewcommand{\cftsecfont}{\rmfamily\mdseries\upshape}
\renewcommand{\cftsecpagefont}{\rmfamily\mdseries\upshape} % No bold!

%%% END Article customizations
\usepackage{xcolor}
\usepackage{tcolorbox}
\usepackage{lipsum}  % 示例文本
\usepackage{mdframed}
\usepackage{pdfpages}


% R code support
\usepackage{listings}
% \lstset{language=R,  % 设置语言为R
%         basicstyle=\ttfamily, % 设置字体样式
%         numbers=left,  % 行号显示在左侧
%         numberstyle=\small\color{blue},  % 行号样式
%         frame=single,  % 设置代码块的边框
%         backgroundcolor=\color{lightgray},  % 设置代码块的背景颜色
%         }

\usepackage{graphicx}
\usepackage{float}
\usepackage{amsmath}





% 设置R语言代码高亮
\lstdefinestyle{rstyle}{
  language=R,
  basicstyle=\ttfamily,
  numbers=left,
  numberstyle=\tiny\color{gray},
  commentstyle=\color{green!40!black},
  keywordstyle=\color{blue},
  stringstyle=\color{purple},
  morekeywords={read.csv},
  frame=single,
  breaklines=true
}
% 

\title{Introduction to Statistics for DS \\ Week 3}
\author{Zehao Qian}
\begin{document}
\maketitle


\section{QA}
The total weight of Quiz's daily breakfast is the sum of the weights of biscuits and meat.

\begin{itemize}
    \item Mean of biscuits = 80g
    \item Mean of meat = 150g
    \item Total mean weight = Mean of biscuits + Mean of meat = 80g + 150g = 230g
\end{itemize}

\paragraph{Since the amounts are independent, the variances add:}

\begin{itemize}
    \item $ Variance\ of\ biscuits = (Standard\ deviation\ of\ biscuits)^2 = 5^2g $
    \item $ Variance\ of\ meat = (Standard\ deviation\ of\ meat)^2 = 8^2g $
\end{itemize}


$$ Total\ variance\ of\ the\ daily\ breakfast \\ = Variance\ of\ biscuits + Variance\ of\ meat \\ = 5^2g + 8^2g = 89g $$


\paragraph{So, the standard deviation of the daily breakfast is the square root of the total variance: $ \sqrt{89}g$}
\section{QB}
$$ 5\ Days\ Mean = 5 * Total mean weight $$
$$ 5\ Days\ standard\ deviation= 25* Total\ variance\ of\ the\ daily\ breakfast$$

\textcolor{red}{\paragraph{\textbf{Prove That:}}
    $$ Var(aX)=a^2 Var(X)$$
    \begin{enumerate}
        \item Start with the definition of variance: $ Var(aX) = E[(aX - E[aX])^2] $
        \item Expand the square and use the linearity of expectations: $ Var(aX) = E[(a^2X^2 - 2aXE[aX] + (E[aX])^2)] $
        \item Use the linearity of expectations to separate the terms: $ Var(aX) = a^2E[X^2] - 2aE[X]E[aX] + (E[aX])^2 $
        \item Since "a" is a constant, E[aX] = aE[X]: $ Var(aX) = a^2E[X^2] - 2a^2E[X]E[X] + (aE[X])^2 $
        \item Simplify: $Var(aX) = a^2E[X^2] - 2a^2(E[X])^2 + a^2(E[X])^2$
        \item Combine like terms: $Var(aX) = a^2E[X^2] - a^2(E[X])^2$
        \item Notice that $a^2$ is a constant, so you can factor it out: $Var(aX) = a^2(E[X^2] - (E[X])^2)$
        \item Recall that $Var(X) = E[X^2] - (E[X])^2$: $Var(aX) = a^2 Var(X)$
    \end{enumerate}}



\paragraph{\textbf{Important Conclusion:}}
\begin{enumerate}
    \item $E(X+Y)=E(X)+E(Y)$
    \item $E(aX)=aE(X)$
    \item $Var(aX)=a^2 Var(X)$
    \item $Var(X+Y)=Var(X)+Var(Y)+2Cov(X,Y)$
    \item $Var(aX+bY)=a^2 Var(X)+b^2 Var(Y)+2ab Cov(X,Y)$
\end{enumerate}

\section{QC}
\paragraph{\textbf{Given:}}
\begin{itemize}
    \item $ \lambda_{weekday\ biscuits}=80 $, $ \sigma_{weekday\ biscuits}=5$
    \item $ \lambda_{weekday\ meat}=150 $, $ \sigma_{weekday\ meat}=8 $
    \item $ \lambda_{weekends\ biscuits}=80 $, $ \sigma_{weekends\ biscuits}=2$
    \item $ \lambda_{weekends\ biscuits}=150 $, $ \sigma_{weekends\ biscuits}=3$
\end{itemize}



\section{QD: Effect on Covariance Between Biscuit Weight and Meat Weight}
\paragraph{The covariance between two random variables X and Y is defined as:}
$$ Cov(X, Y) = E[(X - E[X]) * (Y - E[Y])] $$

\paragraph{The covariance between biscuit weight and meat weight measures the joint variability of these two variables. If you change your measurement method and start measuring less meat when you think you've measured more biscuits, it means that the two variables are no longer independent. This change in measurement approach may lead to a decrease in their covariance.}
\paragraph{The effect on the covariance between biscuit weight and meat weight would depend on how you adapt your measurements based on your perception of biscuit measurements. If your adjustments make the two variables less correlated, the covariance will decrease. If your adjustments make them more correlated, the covariance will increase. The exact effect on the covariance would require a detailed analysis of the measurement adjustments you make.}

\end{document}

